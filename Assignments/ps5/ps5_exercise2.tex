\documentclass[12pt,letterpaper]{cos340hw}


\begin{document}

%%%%%%%%%%%%%%%%%%%%%%%%%%%%%%%%%%%%%%%%%%%%%%%%%%%
%%% HEADER + TITLE %%%%%%%%%%%%%%%%%%%%%%%%%%%%%%%%
\hwPrologue{5}            % Homework
           {2}            % Problem number
           {William Svoboda}  % Your name
           {wsvoboda}   % Your NetID
           %
           % Below, write the collaborators. If no collaborators, type "None"
           {Epi Torres-Smith, Leslie Kim} 
%%%%%%%%%%%%%%%%%%%%%%%%%%%%%%%%%%%%%%%%%%%%%%%%%%%



%%%%%%%%%%%%%%%%%%%%%%%%%%%%%%%%%%%%%%%%%%%%%%%%%%%
%%% YOUR WORK BELOW %%%%%%%%%%%%%%%%%%%%%%%%%%%%%%%

\noindent\textbf{Problem 2:}\\
We wish to show the reduction PERFECT-MATCHING$\le_P$DISJOINT-PATHS.\\\\
We start by considering a simple, bipartite graph $G$ as an input. To transform this input into one for 
the disjoint paths decision problem, we can add two vertices $s$ and $t$ to make the graph $G'$. Each 
vertex in part $A$ will be connected to $s$, and each vertex in part $B$ will be connected to $t$. $G'$ will 
then have $k$ disjoint paths connecting $s$ and $t$ if and only if $G$ is a perfect matching where
$\abs{A}=\abs{B}=k$. If, for example, $G$ did not have a perfect matching, the maximum possible number 
of disjoint paths in $G'$ would be lower; a vertex would exist in $G$ without a matching, so $G'$ would 
have paths that shared the same edge.\\\\
Likewise, if $G'$ has $k$ disjoint paths the added vertices $s$ and $t$ and their edges can be ignored. 
This returns the graph to $G$, which we know must have a perfect matching.
 

%%% END %%%%%%%%%%%%%%%%%%%%%%%%%%%%%%%%%%%%%%%%%%%
%%%%%%%%%%%%%%%%%%%%%%%%%%%%%%%%%%%%%%%%%%%%%%%%%%%

\end{document}
