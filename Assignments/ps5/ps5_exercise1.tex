\documentclass[12pt,letterpaper]{cos340hw}


\begin{document}

%%%%%%%%%%%%%%%%%%%%%%%%%%%%%%%%%%%%%%%%%%%%%%%%%%%
%%% HEADER + TITLE %%%%%%%%%%%%%%%%%%%%%%%%%%%%%%%%
\hwPrologue{5}            % Homework
           {1}            % Problem number
           {William Svoboda}  % Your name
           {wsvoboda}   % Your NetID
           %
           % Below, write the collaborators. If no collaborators, type "None"
           {Epi Torres-Smith, Leslie Kim} 
%%%%%%%%%%%%%%%%%%%%%%%%%%%%%%%%%%%%%%%%%%%%%%%%%%%



%%%%%%%%%%%%%%%%%%%%%%%%%%%%%%%%%%%%%%%%%%%%%%%%%%%
%%% YOUR WORK BELOW %%%%%%%%%%%%%%%%%%%%%%%%%%%%%%%

\noindent\textbf{Problem 1:}\\
We wish to show that 4-colorability is NP-complete. To do so, it is necessary 
to show that it is both in NP and that it is NP-hard.\\\\
\noindent\textbf{1.} To show that 4-colorability is in NP, we can give a 
polynomial time verification algorithm. For each vertex in the graph, all of the adjacent 
vertices are checked and the number of unique colors is tracked. The graph is accepted if exactly four 
colors are used and if no pair of adjacent vertices was found to share the same color. This process takes 
time proportional to $O(n^2)$.\\\\
\noindent\textbf{2.} It is given that 3-colorability is NP-complete. To show that 4-colorability is NP-hard, 
we can show a reduction from the 3-colorability problem to the 4-colorability problem. We start by taking 
an input graph $G$ that is known to be 3-colorable. To transform this input into one for 4-colorability, we
can add a new vertex to $G$ that is connected to every other vertex to make the graph $G'$. Since $G$ 
is 3-colorable, we know that the added vertex must be using a fourth color since it is connected to every 
vertex in $G$ (and by definition of a graph coloring adjacent vertices cannot share the same color). $G'$ 
is then 4-colorable if and only if $G$ is 3-colorable. Because 3-colorability is NP-complete, we know that 4-colorability 
must then be NP-hard.\\\\
Likewise, if $G'$ is 4-colorable we know that $G$ must be 3-colorable. This is because the added vertex in $G'$
is the only vertex of its color, so the vertices in $G$ must be using the remaining three colors.\\\\
Because we have shown that 4-colorability is in NP and is also NP-hard, we know that it is NP-complete.


%%% END %%%%%%%%%%%%%%%%%%%%%%%%%%%%%%%%%%%%%%%%%%%
%%%%%%%%%%%%%%%%%%%%%%%%%%%%%%%%%%%%%%%%%%%%%%%%%%%

\end{document}
