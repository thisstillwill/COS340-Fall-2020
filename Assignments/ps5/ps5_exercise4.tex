\documentclass[12pt,letterpaper]{cos340hw}


\begin{document}

%%%%%%%%%%%%%%%%%%%%%%%%%%%%%%%%%%%%%%%%%%%%%%%%%%%
%%% HEADER + TITLE %%%%%%%%%%%%%%%%%%%%%%%%%%%%%%%%
\hwPrologue{5}            % Homework
           {4}            % Problem number
           {William Svoboda}  % Your name
           {wsvoboda}   % Your NetID
           %
           % Below, write the collaborators. If no collaborators, type "None"
           {Epi Torres-Smith, Leslie Kim} 
%%%%%%%%%%%%%%%%%%%%%%%%%%%%%%%%%%%%%%%%%%%%%%%%%%%



%%%%%%%%%%%%%%%%%%%%%%%%%%%%%%%%%%%%%%%%%%%%%%%%%%%
%%% YOUR WORK BELOW %%%%%%%%%%%%%%%%%%%%%%%%%%%%%%%

\noindent\textbf{Problem 4:}\\
\noindent\textbf{(A):} The algorithm is first called at step $j$ with $a_0$ and $b_0$. During this call, if 
the gcd is not found the algorithm will be called again. At step $j+1$, the value of $a_1 = b_0$ and 
$b_1 = a_0 \mod{b_0}$. If the algorithm is called again, at step $j+1$ we know $a_2$ will have the value 
$b_1 = a_0 \mod{b_0}$ and $b_2 = a_1 \mod{b_1} = b_0 \mod{(a_0 \mod{b_0)}}$.\\\\
We can first observe that the modulo operation will never \emph{increase} the value of $b$. Additionally, 
it is given that $a \ge b > 0$. This means that either $b \le \frac{a}{2}$ or $b > \frac{a}{2}$. If we have 
the case where $b \le \frac{a}{2}$, then in the next call to the algorithm we will know that $a \mod {b} \le b \le \frac{a}{2}$. 
If  $b > \frac{a}{2}$, then $a \mod{b}$ will still be less than or equal to $\frac{a}{2}$. In either case, 
after two calls it will be guaranteed that $b_{j+2} \le \frac{b_j}{2}$.\\\\
\noindent\textbf{(B):} It is given that the binary representation of an integer $x$ is $\log_2x + 1$. We 
can view the algorithm as a series of right shifts to $x$, with each right shift being equal to a division 
by 2. From the last part, we know that $b_{j+2} \le \frac{b_j}{2}$ so there are $\log_2 b_0$ shifts. Since 
this division is guaranteed every two calls, there will therefore be  $2 \cdot \log_2 b_0$ shifts at most. 
In the worst case scenario, the gcd would be equal to 1, so right shifts would need to occur the entire 
length of $b$.


%%% END %%%%%%%%%%%%%%%%%%%%%%%%%%%%%%%%%%%%%%%%%%%
%%%%%%%%%%%%%%%%%%%%%%%%%%%%%%%%%%%%%%%%%%%%%%%%%%%

\end{document}
