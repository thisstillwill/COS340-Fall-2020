\documentclass[12pt,letterpaper]{cos340hw}


\begin{document}

%%%%%%%%%%%%%%%%%%%%%%%%%%%%%%%%%%%%%%%%%%%%%%%%%%%
%%% HEADER + TITLE %%%%%%%%%%%%%%%%%%%%%%%%%%%%%%%%
\hwPrologue{5}            % Homework
           {6}            % Problem number
           {William Svoboda}  % Your name
           {wsvoboda}   % Your NetID
           %
           % Below, write the collaborators. If no collaborators, type "None"
           {Epi Torres-Smith, Leslie Kim} 
%%%%%%%%%%%%%%%%%%%%%%%%%%%%%%%%%%%%%%%%%%%%%%%%%%%



%%%%%%%%%%%%%%%%%%%%%%%%%%%%%%%%%%%%%%%%%%%%%%%%%%%
%%% YOUR WORK BELOW %%%%%%%%%%%%%%%%%%%%%%%%%%%%%%%

\noindent\textbf{Problem 6:}\\
We wish to show that the HITTING-SET decision problem is NP-complete. To do so, it is necessary 
to show that it is both in NP and that it is NP-hard.\\\\
\noindent\textbf{1.} To show that the HITTING-SET decision problem is in NP, we can give a 
polynomial time verification algorithm. For each of the $k$ elements in the candidate hitting set $H$, 
we can go through each of the $m$ subsets of $A$. If a match is found (that is an intersection with one of 
the elements in the current subset), that subset is marked as intersecting. If the number of matches equals 
$m$, then we know that $H$ is a valid hitting set. Because each of the subsets of $A$ can up to $n$ elements 
long, the time complexity is $O(k \cdot n \cdot m)$.\\\\
\noindent\textbf{2.} To show that the HITTING-SET decision problem is NP-hard, we can show the 
reduction VERTEX-COVER$\le_P$HITTING-SET. We start by considering an undirected graph 
$G=(V,E)$. For each edge in $G$, we make a subset containing the two vertices connected by that edge. 
$G$ only has a vertex cover of size $l$ if and only if each of the edge subsets in $G$ has a non-zero intersection 
with the set of vertices that make up the proposed vertex cover. This is equivalent to saying that $G$ has a hitting set 
of size $l$. We have successfully transformed the inputs of the VERTEX-COVER decision problem to those of the 
HITTING-SET decision problem. Because the VERTEX-COVER decision problem is NP-complete, 
we know that the HITTING-SET decision problem must then be NP-hard.\\\\
Likewise, if we had a hitting set of size $k$ for a collection of subsets, with each subset 
representing an edge between two vertices, we could construct a graph by adding an edge from each vertex 
in the hitting set to its intersection with a vertex in the edge subsets. This would give us a valid vertex cover.\\\\
Because we have shown that the HITTING-SET problem is in NP and is also NP-hard, we know that it 
is NP-complete.
 

%%% END %%%%%%%%%%%%%%%%%%%%%%%%%%%%%%%%%%%%%%%%%%%
%%%%%%%%%%%%%%%%%%%%%%%%%%%%%%%%%%%%%%%%%%%%%%%%%%%

\end{document}
