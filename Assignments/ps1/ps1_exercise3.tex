\documentclass[12pt,letterpaper]{cos340hw}


\begin{document}

%%%%%%%%%%%%%%%%%%%%%%%%%%%%%%%%%%%%%%%%%%%%%%%%%%%
%%% HEADER + TITLE %%%%%%%%%%%%%%%%%%%%%%%%%%%%%%%%
\hwPrologue{1}            % Homework
           {3}            % Problem number
           {William Svoboda}  % Your name
           {wsvoboda}   % Your NetID
           %
           % Below, write the collaborators. If no collaborators, type "None"
           {None} 
%%%%%%%%%%%%%%%%%%%%%%%%%%%%%%%%%%%%%%%%%%%%%%%%%%%



%%%%%%%%%%%%%%%%%%%%%%%%%%%%%%%%%%%%%%%%%%%%%%%%%%%
%%% YOUR WORK BELOW %%%%%%%%%%%%%%%%%%%%%%%%%%%%%%%

\noindent\textbf{Problem 3:}\\
We wish to prove that at a party of $n\ge2$ people there exist two people who have the same number
of friends at the party.\\\\
(Base Case) When $n=2$, there are only two people at the party. In the case that neither person has
any friends at the party, by definition they are a pair that has the same number of friends (they 
both have zero friends). In the case that one person has a friend at the party, the predicate is still true
because by definition friendship is symmetric and not reflexive; both people know each other and so
have the same number of friends at the party (both have one friend).\\\\
(Proof) If everyone at the party is friends with at least one other person there, then for $n$ people the
maximum number of friends a single person can have there is $n-1$ since by definition you cannot be 
friends with yourself. This means the set
$$\{1,2,3,\dots, n-1\}$$
which has $n-1$ elements represents the possible numbers of friends at the party a person could have. 
If there is someone at the party who is not friends with anyone, then the set representing the possible
numbers of friends is
$$\{0,1,2,\dots, n-2\}$$
which has $n-1$ elements. The upper bound of $n-2$ in the set comes from there being $n-1$ other
people at the party minus the one person no one is friends with. If there was another "party crasher" 
with no friends at the party, we could stop because such a situation would produce a pair of people who 
both have zero friends there.\\\\
In either case, if there are $n$ people at the party and only $n-1$ possible numbers of friends a person 
could have, then by the Pigeonhole Principle there will be two people with the same number of friends 
at the party.

%%% END %%%%%%%%%%%%%%%%%%%%%%%%%%%%%%%%%%%%%%%%%%%
%%%%%%%%%%%%%%%%%%%%%%%%%%%%%%%%%%%%%%%%%%%%%%%%%%%

\end{document}