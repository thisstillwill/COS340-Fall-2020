\documentclass[12pt,letterpaper]{cos340hw}


\begin{document}

%%%%%%%%%%%%%%%%%%%%%%%%%%%%%%%%%%%%%%%%%%%%%%%%%%%
%%% HEADER + TITLE %%%%%%%%%%%%%%%%%%%%%%%%%%%%%%%%
\hwPrologue{1}            % Homework
           {6}            % Problem number
           {William Svoboda}  % Your name
           {wsvoboda}   % Your NetID
           %
           % Below, write the collaborators. If no collaborators, type "None"
           {None} 
%%%%%%%%%%%%%%%%%%%%%%%%%%%%%%%%%%%%%%%%%%%%%%%%%%%



%%%%%%%%%%%%%%%%%%%%%%%%%%%%%%%%%%%%%%%%%%%%%%%%%%%
%%% YOUR WORK BELOW %%%%%%%%%%%%%%%%%%%%%%%%%%%%%%%

\noindent\textbf{Problem 6:}\\
We wish to prove
$$P(n):\sum_{i=1}^n \frac{1}{(2i-1)(2i+1)}=\frac{n}{2n+1}$$
for all integers $n>0$.\\\\
(Base Case)
We first show $P(1)$ is true, i.e. that the formula is valid when
$n=1$. If $n=1$, then the summation becomes
$$\frac{1}{(2\cdot1-1)(2\cdot1+1)}=\frac{1}{2\cdot1+1}$$
$$\frac{1}{1\cdot3}=\frac{1}{3}$$
$$\frac{1}{3}=\frac{1}{3}$$
and $P(1)$ holds.\\\\
(Inductive Step) Let $k>0$ and suppose that $P(n)$ is true when $n=k$. It now remains to be shown 
following our inductive hypothesis that $P(k+1)$ is true. \\
From our assumption that $P(k)$ is true, we can rewrite the summation for $n=k+1$ as
$$\sum_{i=1}^{k+1} \frac{1}{(2i-1)(2i+1)}=\frac{k}{2k+1}+\frac{1}{(2(k+1)-1)(2(k+1)+1)}$$
We can now simplify the right-hand side as
$$=\frac{k}{2k+1}+\frac{1}{(2k+2-1)(2k+2+1)}$$
$$=\frac{k}{2k+1}+\frac{1}{(2k+1)(2k+3)}$$
$$=\frac{k(2k+3)}{(2k+1)(2k+3)}+\frac{1}{(2k+1)(2k+3)}$$
$$=\frac{k(2k+3)+1}{(2k+1)(2k+3)}$$
$$=\frac{2k^2+3k+1}{(2k+1)(2k+3)}$$
$$=\frac{2k^2+k+2k+1}{(2k+1)(2k+3)}$$
$$=\frac{k(2k+1)+(2k+1)}{(2k+1)(2k+3)}$$
$$=\frac{(2k+1)(k+1)}{(2k+1)(2k+3)}$$
$$=\frac{k+1}{2k+3}$$
This is what we expect for $\frac{n}{2n+1}$ when $n=k+1$, thus $P(k+1)$ is true. By Weak Induction 
we have proven that the formula is valid for all integers $n>0$.

%%% END %%%%%%%%%%%%%%%%%%%%%%%%%%%%%%%%%%%%%%%%%%%
%%%%%%%%%%%%%%%%%%%%%%%%%%%%%%%%%%%%%%%%%%%%%%%%%%%

\end{document}