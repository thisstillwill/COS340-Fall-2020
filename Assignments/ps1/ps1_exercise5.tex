\documentclass[12pt,letterpaper]{cos340hw}


\begin{document}

%%%%%%%%%%%%%%%%%%%%%%%%%%%%%%%%%%%%%%%%%%%%%%%%%%%
%%% HEADER + TITLE %%%%%%%%%%%%%%%%%%%%%%%%%%%%%%%%
\hwPrologue{1}            % Homework
           {5}            % Problem number
           {William Svoboda}  % Your name
           {wsvoboda}   % Your NetID
           %
           % Below, write the collaborators. If no collaborators, type "None"
           {None} 
%%%%%%%%%%%%%%%%%%%%%%%%%%%%%%%%%%%%%%%%%%%%%%%%%%%



%%%%%%%%%%%%%%%%%%%%%%%%%%%%%%%%%%%%%%%%%%%%%%%%%%%
%%% YOUR WORK BELOW %%%%%%%%%%%%%%%%%%%%%%%%%%%%%%%

\noindent\textbf{Problem 5:}\\
We wish to find the minimum number of elements that must be picked from the set $A=\{1,3,5,\dots,99\}$ 
of all the odd positive numbers less than 100 such that there is always a pair of picked numbers that 
sum up to 100.\\\\
(Proof) Intuitively, we know that the possible pairs that sum to $100$ are the $25$ pairs where each 
number in the pair is the same distance from the end of the range $[1,99]$. For example, $A$ can be
split into another set that contains all these pairs:
$$\{(1,99),(3,97),(5,95),\dots,(49,51)\}$$
The Pigeonhole Principle states that if $n$ items are put into $m$ containers and $n>m$ then at least 
one container contains more than one item. If we take $m$ equal to the 25 pairs that sum to 100 we 
can write the inequality
$$n>25$$
where $n$ is the number of integers picked. Thus picking a minimum of $n=26$ integers from $A$ will
guarantee a pair that sum up to 100.

%%% END %%%%%%%%%%%%%%%%%%%%%%%%%%%%%%%%%%%%%%%%%%%
%%%%%%%%%%%%%%%%%%%%%%%%%%%%%%%%%%%%%%%%%%%%%%%%%%%

\end{document}