\documentclass[12pt,letterpaper]{cos340hw}


\begin{document}

%%%%%%%%%%%%%%%%%%%%%%%%%%%%%%%%%%%%%%%%%%%%%%%%%%%
%%% HEADER + TITLE %%%%%%%%%%%%%%%%%%%%%%%%%%%%%%%%
\hwPrologue{1}            % Homework
           {1}            % Problem number
           {William Svoboda}  % Your name
           {wsvoboda}   % Your NetID
           %
           % Below, write the collaborators. If no collaborators, type "None"
           {None} 
%%%%%%%%%%%%%%%%%%%%%%%%%%%%%%%%%%%%%%%%%%%%%%%%%%%



%%%%%%%%%%%%%%%%%%%%%%%%%%%%%%%%%%%%%%%%%%%%%%%%%%%
%%% YOUR WORK BELOW %%%%%%%%%%%%%%%%%%%%%%%%%%%%%%%

\noindent\textbf{Problem 1:}\\
We wish to prove
$$P(n): T_n\le2^n$$
for any nonnegative integer $n$.\\\\
(Base Case) We first show $P(0)$, $P(1)$, and $P(2)$ are true, i.e. that the formula is valid when
$n=0$, $n=1$, and $n=2$. If $n=0$, then
$$T_0\le2^0$$
$$T_0\le1$$
By definition we are given $T_0=1$, so
$$1\le1$$
And $P(0)$ holds. If $n=1$, then
$$T_1\le2^1$$
$$T_1\le2$$
By definition we are given $T_1=1$, so
$$1\le2$$
And $P(1)$ holds. If $n=2$, then
$$T_2\le2^2$$
$$T_2\le4$$
By definition we are given $T_2=1$, so
$$1\le4$$
And $P(2)$ holds.\\\\
(Inductive Step) let $k\ge3$ and suppose that that $P(j)$ is true for all $0\le j\le k$. Since $k\ge3$, then 
$k-3\ge0$ and so we are assuming in particular that $P(k)$, $P(k-1)$, and $P(k-2)$ are all true. It 
now remains to be shown following the inductive hypothesis that $P(k+1)$ is true. \\
From the definition of the sequence we can write
$$T_{k+1}=T_k+T_{k-1}+T_{k-2}$$
Then following from our inductive hypothesis we can write
$$T_{k+1}=T_k+T_{k-1}+T_{k-2}\le2^k+2^{k-1}+2^{k-2}$$
Now the right-hand side of the inequality can be simplified as
$$=\frac{1}{2}\cdot2^{k+1}+\frac{1}{4}\cdot2^{k+1}+\frac{1}{8}\cdot2^{k+1}$$
$$=2^{k+1}\cdot(\frac{1}{2}+\frac{1}{4}+\frac{1}{8})$$
$$=2^{k+1}\cdot(\frac{7}{8})$$
We now have the full inequality written as
$$T_{k+1}\le2^{k+1}$$
Thus $P(k+1)$ is true. By Strong Induction we have proven that the inequality is valid for any 
nonnegative integer $n$.

%%% END %%%%%%%%%%%%%%%%%%%%%%%%%%%%%%%%%%%%%%%%%%%
%%%%%%%%%%%%%%%%%%%%%%%%%%%%%%%%%%%%%%%%%%%%%%%%%%%

\end{document}