\documentclass[12pt,letterpaper]{cos340hw}


\begin{document}

%%%%%%%%%%%%%%%%%%%%%%%%%%%%%%%%%%%%%%%%%%%%%%%%%%%
%%% HEADER + TITLE %%%%%%%%%%%%%%%%%%%%%%%%%%%%%%%%
\hwPrologue{1}            % Homework
           {4}            % Problem number
           {William Svoboda}  % Your name
           {wsvoboda}   % Your NetID
           %
           % Below, write the collaborators. If no collaborators, type "None"
           {None} 
%%%%%%%%%%%%%%%%%%%%%%%%%%%%%%%%%%%%%%%%%%%%%%%%%%%



%%%%%%%%%%%%%%%%%%%%%%%%%%%%%%%%%%%%%%%%%%%%%%%%%%%
%%% YOUR WORK BELOW %%%%%%%%%%%%%%%%%%%%%%%%%%%%%%%

\noindent\textbf{Problem 4:}\\
We wish to prove
$$P(n): \textrm{There are always two sums that are equal}$$
out of the row, column, and diagonal sums for a $n\times n$ board.\\\\
(Base Case) We first show $P(1)$ is true, i.e. that the formula is valid when $n=1$. If $n=1$, then the
table is made up of only a single entry. Because there is only one entry, the rows, columns, and 
diagonals of the table are the same (they all cover only the single square). No matter what value the 
entry has the sums will be equal and $P(1)$ holds.\\\\
(Proof) There will always be $2n+2$ sums from an $n\times n$ board. This can be concluded from the
observation that the two available diagonals is constant for all square boards, and that there are $n$ 
rows and $n$ columns.\\\\
We wish to choose two sums out of the $2n+2$ available sums from the board that are equal. This can
be represented by
$$\binom{(2n+2)+2-1}{2}=\binom{2n+3}{2}$$
Because repetitions are allowed and the order is not important.

%%% END %%%%%%%%%%%%%%%%%%%%%%%%%%%%%%%%%%%%%%%%%%%
%%%%%%%%%%%%%%%%%%%%%%%%%%%%%%%%%%%%%%%%%%%%%%%%%%%

\end{document}