\documentclass[12pt,letterpaper]{cos340hw}


\begin{document}

%%%%%%%%%%%%%%%%%%%%%%%%%%%%%%%%%%%%%%%%%%%%%%%%%%%
%%% HEADER + TITLE %%%%%%%%%%%%%%%%%%%%%%%%%%%%%%%%
\hwPrologue{2}            % Homework
           {4}            % Problem number
           {William Svoboda}  % Your name
           {wsvoboda}   % Your NetID
           %
           % Below, write the collaborators. If no collaborators, type "None"
           {Epi Torres-Smith, Leslie Kim} 
%%%%%%%%%%%%%%%%%%%%%%%%%%%%%%%%%%%%%%%%%%%%%%%%%%%



%%%%%%%%%%%%%%%%%%%%%%%%%%%%%%%%%%%%%%%%%%%%%%%%%%%
%%% YOUR WORK BELOW %%%%%%%%%%%%%%%%%%%%%%%%%%%%%%%

\noindent\textbf{Problem 4:}\\
We wish to prove or disprove that rolling two sixes if you know the first roll is a six is at least as likely as 
rolling two sixes if you know that at least one of the rolls is a six.\\\\
Let $P(A)$ be the probability the first roll is a six, and let $P(B)$ be the probability that the second roll is 
a six. This means the probability of rolling two sixes is $P(A\cap B)$, because intuitively the intersection 
where both the first roll is a six and the second roll is a six is just where both rolls are sixes. We can also 
say that the probability of one of the rolls being a six is $P(A \cup B)$ since by the inclusion-exclusion 
principle the union of two sets is their sum minus their intersection (with the intersection just being where both 
rolls produce sixes).\\\\
We can then write the inequality as
$$P(A\cap B|A)\ge P(A\cap B|A \cup B)$$
Using the formula for conditional probability, we can expand the terms on each side:
$$\frac{P((A\cap B)\cap A)}{P(A)}\ge\frac{P((A\cap B)\cap (A\cup B))}{P(A\cup B)}$$
On the left side, we know that $(A\cap B)\cap A$ is still $A\cap B$ and that on the right side that 
$(A\cap B)\cap (A\cup B)$ is still $A\cap B$ so we have that
\begin{align*}
\frac{P(A\cap B)}{P(A)} & \ge\frac{P(A\cap B))}{P(A\cup B)}\\
\frac{1}{P(A)} & \ge\frac{1}{P(A\cup B)}
\end{align*}
We can now see clearly that the left hand event (the probability of rolling two sixes given the first roll is 
a six) is at least as likely as rolling two sixes given one of the rolls is a six. $A\cup B$ covers the entire 
area that $A$ and $B$ do, which is larger than $A$ alone. This means that $\frac{1}{P(A)}$ is greater 
than $\frac{1}{P(A\cup B)}$. At worst, if $A$ and $B$ intersect completely it would simply mean that 
$\frac{1}{P(A)}=\frac{1}{P(A\cup B)}$.

%%% END %%%%%%%%%%%%%%%%%%%%%%%%%%%%%%%%%%%%%%%%%%%
%%%%%%%%%%%%%%%%%%%%%%%%%%%%%%%%%%%%%%%%%%%%%%%%%%%

\end{document}
