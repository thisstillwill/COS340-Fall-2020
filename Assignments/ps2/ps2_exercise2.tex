\documentclass[12pt,letterpaper]{cos340hw}


\begin{document}

%%%%%%%%%%%%%%%%%%%%%%%%%%%%%%%%%%%%%%%%%%%%%%%%%%%
%%% HEADER + TITLE %%%%%%%%%%%%%%%%%%%%%%%%%%%%%%%%
\hwPrologue{2}            % Homework
           {2}            % Problem number
           {William Svoboda}  % Your name
           {wsvoboda}   % Your NetID
           %
           % Below, write the collaborators. If no collaborators, type "None"
           {Epi Torres-Smith, Leslie Kim} 
%%%%%%%%%%%%%%%%%%%%%%%%%%%%%%%%%%%%%%%%%%%%%%%%%%%



%%%%%%%%%%%%%%%%%%%%%%%%%%%%%%%%%%%%%%%%%%%%%%%%%%%
%%% YOUR WORK BELOW %%%%%%%%%%%%%%%%%%%%%%%%%%%%%%%

\noindent\textbf{Problem 2:}\\
\noindent\textbf{A:}\\
We wish to find the number of ways that $n\ge3$ balls can be distributed in $m\ge2$ boxes when the 
balls are identical.\\\\
We can start by thinking of each configuration as a bit vector in the form $BB||BBB$, where $B$ represents a 
ball and $|$ the space between two consecutive boxes. With the fewest number of balls and boxes one configuration 
would look like $BB|B$. We can observe that for $m$ boxes there are only $m-1$ box dividers in the bit vector. The 
total length of the bit vector would then be $n+m-1$ which is the number of balls plus the number of dividers. Then if 
we wish to find the number of ways that we can choose $n$ balls out of the $n+m-1$ total configuration length we have
$$\binom{n+m-1}{n}$$
\noindent\textbf{B:}\\
We still wish to find the number of ways we can distribute the balls, but now not all balls need to be distributed. This means
the length of the bit vector for each configuration is now $\le n+m-1$. As a result, we can choose \emph{up} to $n$ balls 
out of the configuration length that results from the number of balls picked to distribute. The number of ways can then be expressed
with the summation
$$\sum_{k=0}^{k=n}\binom{k+m-1}{k}$$
\noindent\textbf{C:}\\
In this situation we know there are still $n\ge3$ identical balls that need to be distributed, but the second box is guaranteed to take at least 3 of
those balls. The smallest configuration in this case would look like $|BBB$. We have 3 less balls balls to distribute now, so we are picking
$n-3$ balls out of a bit vector of length $n+m-1-3$ to guarantee that the first three balls are accounted for. The number of ways to do this is
$$\binom{n+m-4}{n-3}$$
\noindent\textbf{D:}\\
In this case the balls are pairwise different and the order that they are placed matters. When order matters and there should be no repetitions, 
there are $\frac{n!}{(n-k)!}$ ways to choose $k$ out of $n$ objects. We are still choosing $n$ balls out of a configuration length of $n+m-1$, 
so the number of ways to do so is
$$\frac{(n+m-1)!}{(n+m-1-n)}=\frac{(n+m-1)!}{(m-1)}$$
\noindent\textbf{E:}\\
Like the previous case, the balls are pairwise different and the order they are placed matters. However,  only the balls 1, 2, 3 are placed in 
the box with number 1. This means the total number of ways to distribute the balls is equal to the number of ways to distribute 3 balls into a 
single box times the number of ways to distribute the other balls among the other boxes. The first action can be accomplished in $3!$ ways. 
This means we now need to distribute 3 fewer balls among 1 fewer boxes. The length of the bit vector needs to be $n+m-1-3-1$ to account for 
having one less box to work with and 3 fewer balls to distribute. We then choose $n-3$ balls out of the configuration length, or
$$\frac{(n+m-1-3-1)!}{(n+m-1-3-1-n-3)!}=\frac{(n+m-1-3-1)!}{(m-2)!}$$
So that the total number of ways to distribute the balls is
$$3!\cdot\frac{(n+m-5)!}{(m-2)!}$$

%%% END %%%%%%%%%%%%%%%%%%%%%%%%%%%%%%%%%%%%%%%%%%%
%%%%%%%%%%%%%%%%%%%%%%%%%%%%%%%%%%%%%%%%%%%%%%%%%%%

\end{document}
