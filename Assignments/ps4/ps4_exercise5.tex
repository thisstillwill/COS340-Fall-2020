\documentclass[12pt,letterpaper]{cos340hw}


\begin{document}

%%%%%%%%%%%%%%%%%%%%%%%%%%%%%%%%%%%%%%%%%%%%%%%%%%%
%%% HEADER + TITLE %%%%%%%%%%%%%%%%%%%%%%%%%%%%%%%%
\hwPrologue{4}            % Homework
           {5}            % Problem number
           {William Svoboda}  % Your name
           {wsvoboda}   % Your NetID
           %
           % Below, write the collaborators. If no collaborators, type "None"
           {Epi Torres-Smith, Leslie Kim} 
%%%%%%%%%%%%%%%%%%%%%%%%%%%%%%%%%%%%%%%%%%%%%%%%%%%



%%%%%%%%%%%%%%%%%%%%%%%%%%%%%%%%%%%%%%%%%%%%%%%%%%%
%%% YOUR WORK BELOW %%%%%%%%%%%%%%%%%%%%%%%%%%%%%%%

\noindent\textbf{Problem 5:}\\
We wish to prove that the number of vertices in a graph is at most the product of the independence 
number and the chromatic number.\\\\
By the definition of a legal graph coloring, adjacent vertices will be of different colors. This means that 
all vertices of the same color will be in an independent set, since they would not have edges in common. 
If they did, then the graph coloring would be invalid since it would mean that adjacent vertices could be 
the same color.\\\\
It is also given that the independence number of a graph is the size of the maximum independent set of 
the graph. Since each color corresponds to an independent set, we know that the size of each independent 
set must be less than or equal to the independence number. Therefore, if the product of the independence number 
and the chromatic number is the size of the largest independent set times the number of colorings in the graph, 
we know that this product will always be greater than or equal to the sum of each independent set's size (which 
is equal to the total number of vertices). It would be impossible for the number of vertices to be larger than this 
product, because by definition the size of each set is bounded by the independence number.\\\\
Therefore, the number of vertices in a graph is at most the product of the independence number and the 
chromatic number.


%%% END %%%%%%%%%%%%%%%%%%%%%%%%%%%%%%%%%%%%%%%%%%%
%%%%%%%%%%%%%%%%%%%%%%%%%%%%%%%%%%%%%%%%%%%%%%%%%%%

\end{document}
