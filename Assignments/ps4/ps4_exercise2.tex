\documentclass[12pt,letterpaper]{cos340hw}


\begin{document}

%%%%%%%%%%%%%%%%%%%%%%%%%%%%%%%%%%%%%%%%%%%%%%%%%%%
%%% HEADER + TITLE %%%%%%%%%%%%%%%%%%%%%%%%%%%%%%%%
\hwPrologue{4}            % Homework
           {2}            % Problem number
           {William Svoboda}  % Your name
           {wsvoboda}   % Your NetID
           %
           % Below, write the collaborators. If no collaborators, type "None"
           {Epi Torres-Smith, Leslie Kim} 
%%%%%%%%%%%%%%%%%%%%%%%%%%%%%%%%%%%%%%%%%%%%%%%%%%%



%%%%%%%%%%%%%%%%%%%%%%%%%%%%%%%%%%%%%%%%%%%%%%%%%%%
%%% YOUR WORK BELOW %%%%%%%%%%%%%%%%%%%%%%%%%%%%%%%

\noindent\textbf{Problem 2:}\\
We wish to prove that the chromatic number of an interval graph is equal to its clique number.\\\\
To begin, we will use a greedy algorithm to color the vertices of the graph such that each vertex is given 
the first available color. The algorithm starts by selecting the first interval and giving that vertex one color. 
The next interval is then considered, and if it intersects with the first interval it is given a different color. 
In general, each interval in the graph will be assigned the first available color that has not already been 
used in any of the other intervals that it intersects with. Intersections are found for a given interval by checking 
each of the other intervals in the graph.\\\\
With the graph colored, the first step is to prove that the chromatic number is at least as large as the clique 
number. We can do this by contradiction, with the assumption that the chromatic number is actually less 
than the clique number. It is given that the clique number of a graph is the number of vertices in the 
largest complete subgraph of the graph. For a complete subgraph to exist in a legal coloring, each vertex 
in the subgraph would need to be a different color. However, if the chromatic number is less than the 
clique number it would would mean the total number of colors used is less than the size of the largest complete 
subgraph. This is a contradiction, because it would allow adjacent vertices in the subgraph to share the same 
color. Therefore, the chromatic number is at least as large as the clique number.\\\\
The second step is to prove that the clique number is at least as large as the chromatic number. We can 
do this by contradiction, with the assumption that the clique number is actually smaller than the chromatic 
number. The size of the largest subgraph determines the minimum number of colors needed, 
because nowhere else in the graph contains more vertices that are adjacent to each other. If the clique number is smaller, 
it means that there are excess colors that are not used in the largest subgraph. This is a contradiction, because 
by definition the chromatic number of a graph is the minimal number of colors in which the graph can be 
colored. Therefore, the clique number is at least as large as the chromatic number.\\\\
Because the chromatic number is at least as large as the clique number, and the clique number is at least 
as large as the chromatic number, the chromatic number of an interval graph must equal its clique number.


%%% END %%%%%%%%%%%%%%%%%%%%%%%%%%%%%%%%%%%%%%%%%%%
%%%%%%%%%%%%%%%%%%%%%%%%%%%%%%%%%%%%%%%%%%%%%%%%%%%

\end{document}
