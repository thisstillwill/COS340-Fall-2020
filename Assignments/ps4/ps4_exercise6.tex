\documentclass[12pt,letterpaper]{cos340hw}


\begin{document}

%%%%%%%%%%%%%%%%%%%%%%%%%%%%%%%%%%%%%%%%%%%%%%%%%%%
%%% HEADER + TITLE %%%%%%%%%%%%%%%%%%%%%%%%%%%%%%%%
\hwPrologue{4}            % Homework
           {6}            % Problem number
           {William Svoboda}  % Your name
           {wsvoboda}   % Your NetID
           %
           % Below, write the collaborators. If no collaborators, type "None"
           {Epi Torres-Smith, Leslie Kim} 
%%%%%%%%%%%%%%%%%%%%%%%%%%%%%%%%%%%%%%%%%%%%%%%%%%%



%%%%%%%%%%%%%%%%%%%%%%%%%%%%%%%%%%%%%%%%%%%%%%%%%%%
%%% YOUR WORK BELOW %%%%%%%%%%%%%%%%%%%%%%%%%%%%%%%

\noindent\textbf{Problem 6:}\\
We wish to show that for any possible valid placement of the pebbles, there is a subset of $2n$ pebbles 
such that the row numbers are not the same and the column numbers are not the same for any pair of those 
$2n$ pebbles.\\\\
We can picture the board as a graph with $2n$ vertices representing the columns and $2n$ vertices representing 
the rows. Edges between row and column vertices represent the squares where pebbles are placed. 
It is not possible to have an edge between two column vertices or two row vertices, because otherwise 
pebbles could overlap multiple columns or multiple rows.\\\\
Let the set $L$ represent the $2n$ column vertices and let the set $R$ represent the $2n$ row vertices, such 
that the union of $L$ and $R$ gives the total $4n^2$ vertices in the graph. By definition, this is a bipartite 
graph. Let $S \subseteq L$, and let $N(S) \subseteq R$ be the set of neighbors of all the vertices in $S$.\\\\
By Hall’s Marriage Theorem, there exists a perfect matching in the graph if and only if for
every $S \subseteq L$ we also have $\abs{N(S)} \ge \abs{S}$.\\\\
We know that the number of edges from $S$ to $N(S)$ is equal to the size of $S$ times the degree of its 
vertices. The number of edges from $N(S)$ to $L$ is likewise equal to the size of $N(S)$ times the degree 
of its vertices. Assume that $\abs{N(S)} < \abs{S}$. If this were true, the number of edges from $S$ would be greater than that 
coming from $N(S)$. This is a contradiction, however, because we know that $S \subseteq L$ and so cannot 
actually have more edges than the number coming from $N(S)$ to $L$. Therefore, we have that 
$\abs{N(S)} \ge \abs{S}$ and by Hall's Marriage Theorem there is a perfect matching in the graph.\\\\
If there is a perfect matching, then by definition the degree of each vertex is exactly 1. Therefore, accounting 
for the  $2n$ columns and $2n$ rows there must be a subset of $2n$ pebbles such that the row numbers 
are not the same and the column numbers are not the same for any pair of those $2n$ pebbles.

%%% END %%%%%%%%%%%%%%%%%%%%%%%%%%%%%%%%%%%%%%%%%%%
%%%%%%%%%%%%%%%%%%%%%%%%%%%%%%%%%%%%%%%%%%%%%%%%%%%

\end{document}
