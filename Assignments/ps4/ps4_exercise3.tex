\documentclass[12pt,letterpaper]{cos340hw}


\begin{document}

%%%%%%%%%%%%%%%%%%%%%%%%%%%%%%%%%%%%%%%%%%%%%%%%%%%
%%% HEADER + TITLE %%%%%%%%%%%%%%%%%%%%%%%%%%%%%%%%
\hwPrologue{4}            % Homework
           {3}            % Problem number
           {William Svoboda}  % Your name
           {wsvoboda}   % Your NetID
           %
           % Below, write the collaborators. If no collaborators, type "None"
           {Epi Torres-Smith, Leslie Kim} 
%%%%%%%%%%%%%%%%%%%%%%%%%%%%%%%%%%%%%%%%%%%%%%%%%%%



%%%%%%%%%%%%%%%%%%%%%%%%%%%%%%%%%%%%%%%%%%%%%%%%%%%
%%% YOUR WORK BELOW %%%%%%%%%%%%%%%%%%%%%%%%%%%%%%%

\noindent\textbf{Problem 3:}\\
\noindent\textbf{(A):} We wish to find the number of ways we can color the vertices of a complete graph 
$K_n$, where $n \ge 1$, when provided with $k$ different colors such that $k \ge n$.\\\\
In a complete graph, each pair of vertices is connected by an edge. By the definition of a legal coloring, 
each vertex in the complete graph must be a different color. Otherwise, a pair of adjacent vertices would 
exist that share the same color. The specific coloring depends on which vertices are given which colors, 
so order matters. In addition, there should be no repetition because the number of colors is at least as large as 
the number of vertices. Therefore, because we are choosing how to assign $n$ vertices to $k$ colors, the number 
of ways to color the graph is equal to
$$\frac{k!}{(k-n)!}$$
\noindent\textbf{(B):} We wish to show that the number of ways to color the tree $T_n$, with $n \ge 1$, is equal 
to $k \cdot (k-1)^{n-1}$ when provided with $k \ge 2$ different colors.\\\\
We can use an inductive approach. In the base case, where $n=1$, there is only a single vertex. With $k$ available 
colors, there are simply $k$ ways to color this single vertex. This is equivalent to the value of 
$k \cdot (k-1)^{1-1}=k$, so the predicate holds. If there are two vertices, than the root vertex can still be 
colored in $k$ ways. By the definition of a tree, the second vertex only has one parent. Because the parent is 
already using one of the available colors, the child vertex can only pick from among the remaining $k-1$ colors. 
This means there are $k \cdot (k-1)$ ways to color the two vertices, which is equivalent to $k \cdot (k-1)^{n-1}$ 
when $n=2$:
$$k \cdot (k-1)^{2-1}=k \cdot (k-1)$$
Let $k \cdot (k-1)^{n-2}$ represent the number of ways to color the tree at the $j$-th step before adding an additional 
vertex to the tree, and assume that the predicate $P(j)$ holds for $n \ge 1$ vertices and $j=n$. At the 
$j+1$-th step, an additional vertex is added to the tree. By the definition of a tree, no matter where this vertex is added it 
will have an edge to exactly one parent. Because this parent vertex is already using one of the available colors, 
the added vertex can only pick from among the remaining $k-1$ colors. If there are $k \cdot (k-1)^{n-2}$ ways to 
color the graph at the $j$-th step, then the number of ways to color the graph at the $j+1$-th step is
$$k \cdot (k-1)^{n-2} \cdot (k-1) = k \cdot (k-1)^{n-1}$$
By weak induction, we have shown that the number of ways to color the graph is $k \cdot (k-1)^{n-1}$.


%%% END %%%%%%%%%%%%%%%%%%%%%%%%%%%%%%%%%%%%%%%%%%%
%%%%%%%%%%%%%%%%%%%%%%%%%%%%%%%%%%%%%%%%%%%%%%%%%%%

\end{document}
