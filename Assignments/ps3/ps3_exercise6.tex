\documentclass[12pt,letterpaper]{cos340hw}


\begin{document}

%%%%%%%%%%%%%%%%%%%%%%%%%%%%%%%%%%%%%%%%%%%%%%%%%%%
%%% HEADER + TITLE %%%%%%%%%%%%%%%%%%%%%%%%%%%%%%%%
\hwPrologue{3}            % Homework
           {6}            % Problem number
           {William Svoboda}  % Your name
           {wsvoboda}   % Your NetID
           %
           % Below, write the collaborators. If no collaborators, type "None"
           {Epi Torres-Smith, Leslie Kim} 
%%%%%%%%%%%%%%%%%%%%%%%%%%%%%%%%%%%%%%%%%%%%%%%%%%%



%%%%%%%%%%%%%%%%%%%%%%%%%%%%%%%%%%%%%%%%%%%%%%%%%%%
%%% YOUR WORK BELOW %%%%%%%%%%%%%%%%%%%%%%%%%%%%%%%

\noindent\textbf{Problem 6:}\\
We wish to show that a simple graph is 2-edge-connected if and only if it is connected and every edge 
of the graph is traversed by a cycle.\\\\
First, let us assume that we have a 2-edge-connected simple graph that is \emph{disconnected}. This 
is a contradiction, because by definition a $k$-edge-connected graph must be connected.\\\\
Let us then assume that we have a 2-edge-connected simple graph where at least one edge is 
\emph{not} traversed by a cycle. By definition, a cycle is a path where the first and last vertices are 
repeated (creating a loop). If there is an edge not traversed by a cycle, then it is possible to remove 
that single edge and create a disconnection in that part of the graph (since the vertices that covered 
that edge would no longer be in the path). This is also a contradiction, since by definition it should be 
possible to remove $2-1=1$ edge from a 2-edge-connected graph without disconnecting it.\\\\
Therefore, a simple graph is 2-edge-connected if and only if it is connected and every edge 
of the graph is traversed by a cycle.

%%% END %%%%%%%%%%%%%%%%%%%%%%%%%%%%%%%%%%%%%%%%%%%
%%%%%%%%%%%%%%%%%%%%%%%%%%%%%%%%%%%%%%%%%%%%%%%%%%%

\end{document}
