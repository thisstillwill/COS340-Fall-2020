\documentclass[12pt,letterpaper]{cos340hw}


\begin{document}

%%%%%%%%%%%%%%%%%%%%%%%%%%%%%%%%%%%%%%%%%%%%%%%%%%%
%%% HEADER + TITLE %%%%%%%%%%%%%%%%%%%%%%%%%%%%%%%%
\hwPrologue{3}            % Homework
           {5}            % Problem number
           {William Svoboda}  % Your name
           {wsvoboda}   % Your NetID
           %
           % Below, write the collaborators. If no collaborators, type "None"
           {Epi Torres-Smith, Leslie Kim} 
%%%%%%%%%%%%%%%%%%%%%%%%%%%%%%%%%%%%%%%%%%%%%%%%%%%



%%%%%%%%%%%%%%%%%%%%%%%%%%%%%%%%%%%%%%%%%%%%%%%%%%%
%%% YOUR WORK BELOW %%%%%%%%%%%%%%%%%%%%%%%%%%%%%%%

\noindent\textbf{Problem 5:}\\
We wish to prove $P(n)$, that a simple graph $G$ with $2n$ nodes with no triangles has at most $n^2$ 
edges for $n\ge1$.\\\\
(Base Case) We first show that the predicate holds when $n=1$. When $n=1$, we have $2\cdot1=2$ 
nodes. The first base case is when the two nodes are disconnected, so the total number of edges is 0. 
This is less than or equal to the maximum of $n^2 =1^2=1$ edge. The second base case is when the 
two nodes have a single edge between them. The total number of edges is 1, which is still less than or 
equal to the allowed maximum. No more edges could be added between the two vertices since $G$ is 
a simple graph and no self loops or multiple edges are allowed.\\\\
(Inductive Step) Let $n=k$ and suppose that $P(k)$ is true for $n\ge1$. It now remains to be shown 
following the inductive hypothesis that $P(k+1)$ is true.\\\\
Let the graph $G'$ represent the case for $P(k)$ where there are $2k$ nodes and a maximum of $k^2$ 
edges, and let the graph $G$ represent the case for $P(k+1)$ where there are $2(k+1)=2k+2$ nodes and 
a maximum of $(k+1)^2$ edges. We start by removing two nodes from $G$ such that there are now $2k$ 
nodes left and any edge connected to those two nodes is also removed. This leaves us with $G'$, which 
we assume (from our inductive hypothesis) does not have any cycles of length 3.\\\\
We then add back the two removed nodes. The two nodes can have an edge between them, adding an 
additional edge to the maximum of $k^2$ edges for $G'$. Each of the $2k$ original nodes in $G'$ can be 
connected to either one or the other of the added nodes. This is because if one of the $2k$ nodes has 
edges to both of the added nodes, and the added nodes are also neighbors, a triangle would be formed. 
This means there is a total of $2k$ edges from the $2k$ nodes in $G'$ to the additional two nodes. The 
maximum number of edges is then $k^2+2k+1=(k+1)^2$, which is the bound for the $2(k+1)$ nodes in $G$.\\\\
By weak induction, we have proven that $G$ can have at most $n^2$ edges for $2n$ nodes without 
having any triangles for $n\ge1$.


%%% END %%%%%%%%%%%%%%%%%%%%%%%%%%%%%%%%%%%%%%%%%%%
%%%%%%%%%%%%%%%%%%%%%%%%%%%%%%%%%%%%%%%%%%%%%%%%%%%

\end{document}
