\documentclass[12pt,letterpaper]{cos340hw}


\begin{document}

%%%%%%%%%%%%%%%%%%%%%%%%%%%%%%%%%%%%%%%%%%%%%%%%%%%
%%% HEADER + TITLE %%%%%%%%%%%%%%%%%%%%%%%%%%%%%%%%
\hwPrologue{3}            % Homework
           {3}            % Problem number
           {William Svoboda}  % Your name
           {wsvoboda}   % Your NetID
           %
           % Below, write the collaborators. If no collaborators, type "None"
           {Epi Torres-Smith, Leslie Kim} 
%%%%%%%%%%%%%%%%%%%%%%%%%%%%%%%%%%%%%%%%%%%%%%%%%%%



%%%%%%%%%%%%%%%%%%%%%%%%%%%%%%%%%%%%%%%%%%%%%%%%%%%
%%% YOUR WORK BELOW %%%%%%%%%%%%%%%%%%%%%%%%%%%%%%%

\noindent\textbf{Problem 3:}\\
We wish to show that any simple graph $G=(V,E)$ with $\delta(G)>\frac{1}{2}\cdot(\abs{V}-2)$ is 
connected.\\\\
We can partition $G$ into two subgraphs with $\frac{\abs{V}}{2}$ vertices each. If we take 
any vertex $v$ from either of the subgraphs, the maximum degree it could have is 
$\frac{\abs{V}}{2}-1$ since it could be connected to every vertex in the subgraph besides 
itself (creating a connected component). There could not be more edges from the vertex because in a 
simple graph self loops and multiple edges are not allowed. However, it is given that 
$\delta(G)>\frac{1}{2}\cdot(\abs{V}-2)=\frac{\abs{V}}{2}-1$, so there must actually be one additional edge 
from $v$ to the other connected component. Therefore, the graph $G$ must actually be connected.


%%% END %%%%%%%%%%%%%%%%%%%%%%%%%%%%%%%%%%%%%%%%%%%
%%%%%%%%%%%%%%%%%%%%%%%%%%%%%%%%%%%%%%%%%%%%%%%%%%%

\end{document}
