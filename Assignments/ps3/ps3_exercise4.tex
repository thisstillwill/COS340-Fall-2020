\documentclass[12pt,letterpaper]{cos340hw}


\begin{document}

%%%%%%%%%%%%%%%%%%%%%%%%%%%%%%%%%%%%%%%%%%%%%%%%%%%
%%% HEADER + TITLE %%%%%%%%%%%%%%%%%%%%%%%%%%%%%%%%
\hwPrologue{3}            % Homework
           {4}            % Problem number
           {William Svoboda}  % Your name
           {wsvoboda}   % Your NetID
           %
           % Below, write the collaborators. If no collaborators, type "None"
           {Epi Torres-Smith, Leslie Kim} 
%%%%%%%%%%%%%%%%%%%%%%%%%%%%%%%%%%%%%%%%%%%%%%%%%%%



%%%%%%%%%%%%%%%%%%%%%%%%%%%%%%%%%%%%%%%%%%%%%%%%%%%
%%% YOUR WORK BELOW %%%%%%%%%%%%%%%%%%%%%%%%%%%%%%%

\noindent\textbf{Problem 4:}\\
We wish to determine if there is an $\ell$-regular graph $G=(V,E)$ where $\ell$ is an odd positive integer 
and $\abs{V}$ is also an odd positive integer for $\ell \ge 1$.\\\\
The handshaking theorem states that 
$$\sum_{v\in V}deg(V)=2\cdot \abs{E}$$
If we had both an odd positive number of vertices and all vertices had degree $\ell$ that was also 
odd positive, then the summation $\sum_{v\in V}deg(V)$ would be odd (an odd number added 
odd times is still odd). However, we know that this summation must be equal to $2\cdot \abs{E}$ which 
is always an even positive number.\\\\
This is a contradiction, so by the handshaking theorem there is no $\ell$-regular graph $G=(V,E)$ with 
$\abs{V}$ an odd positive integer for $\ell \ge 1$.


%%% END %%%%%%%%%%%%%%%%%%%%%%%%%%%%%%%%%%%%%%%%%%%
%%%%%%%%%%%%%%%%%%%%%%%%%%%%%%%%%%%%%%%%%%%%%%%%%%%

\end{document}
